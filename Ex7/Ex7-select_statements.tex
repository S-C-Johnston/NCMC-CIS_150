%&tex
\documentclass{article}

\usepackage{graphicx}[draft]
\graphicspath{{img/}}
\usepackage{geometry}
\usepackage{hyperref}
\usepackage{subcaption}
\usepackage{float}
\usepackage{listings}

\title{Exercise 5 -- SQL server install}
\author{Stewart Johnston\\
  {CIS 150 -- Intro to Database Administration}\\
  {NCMC}\\
  {\texttt{johnstons1@student.ncmich.edu}}
}
\date{\today}

\begin{document}
\lstset{language=SQL}

\maketitle

\tableofcontents
\listoffigures

\section{SELECT four columns}

\begin{lstlisting}
SELECT ProductCode
,ProductName
,ListPrice
,DiscountPercent
FROM Products
ORDER BY ListPrice DESC
\end{lstlisting}\label{sec:q_1}
See figure \ref{fig:q_1}

\begin{figure}[H]\centering
	\caption{Select four columns of the products table, per section
	\ref{sec:q_1}}
	\includegraphics[width=\linewidth]{img/q_1}
	\label{fig:q_1}
\end{figure}

\section{Return FullName}

\begin{lstlisting}
SELECT LastName + ', ' + FirstName AS FullName
FROM Customers
ORDER BY LastName
\end{lstlisting}\label{sec:q_2}
See figure \ref{fig:q_2}

\begin{figure}[H]\centering
	\caption{Summarized FullName, as per \ref{sec:q_2}}
	\includegraphics[width=\linewidth]{q_2}
	\label{fig:q_2}
\end{figure}

\section{Select name, listprice, dateadded}

\begin{lstlisting}
SELECT ProductName
,ListPrice
,DateAdded
FROM Products
WHERE ListPrice BETWEEN 500 AND 2000
ORDER BY DateAdded DESC
\end{lstlisting}\label{sec:q_3}
See figure \ref{fig:q_3}

\begin{figure}[H]\centering
	\caption{Returns Name, Price, and Date, per \ref{sec:q_3}}
	\includegraphics[width=\linewidth]{q_3}
	\label{fig:q_3}
\end{figure}

\section{Return Name, Price, Discount, etc}

\begin{lstlisting}
SELECT ProductName
,ListPrice
,DiscountPercent
,ListPrice * (DiscountPercent / 100 ) AS DiscountAmount
,ListPrice - ListPrice * (DiscountPercent / 100 ) AS DiscountPrice /*Because
aliases aren't reusable in the same scope, much to my annoyance, and we haven't
learned subqueries or CTEs yet.*/
FROM Products
ORDER BY DiscountPrice DESC
\end{lstlisting}\label{sec:q_4}
See figure \ref{fig:q_4}

\begin{figure}[H]\centering
	\caption{Per \ref{sec:q_4}, return Name, Price, and Discount, ordered
	by DiscountPrice}
	\includegraphics[width=\linewidth]{q_4}
	\label{fig:q_4}
\end{figure}

\section{Return Item ID, Price, Discount, etc from OrderItems}

\begin{lstlisting}
SELECT ItemID
,ItemPrice
,DiscountAmount
,Quantity
,ItemPrice * Quantity AS PriceTotal
,DiscountAmount * Quantity AS DiscountTotal
,(ItemPrice - DiscountAmount) * Quantity AS ItemTotal
FROM OrderItems
WHERE ((ItemPrice - DiscountAmount) * Quantity) > 500 /*Again, no subqueries
yet. Kind of dumb that this is how it works by default, because otherwise, I'd
happily use ItemTotal*/
ORDER BY ItemTotal DESC
\end{lstlisting}\label{sec:q_5}
See figure \ref{fig:q_5}

\begin{figure}[H]\centering
	\caption{Per \ref{sec:q_5}, return ID, Price, etc from OrderItems}
	\includegraphics[width=\linewidth]{q_5}
	\label{fig:q_5}
\end{figure}

\section{Return attributes where one is null}

\begin{lstlisting}
SELECT OrderID
,OrderDate
,ShipDate
FROM Orders
WHERE ShipDate is null
\end{lstlisting}\label{sec:q_6}
See figure \ref{fig:q_6}

\begin{figure}[H]\centering
	\caption{Per \ref{sec:q_6}, return ID, Dates, from Orders, where
	ShipDate is null}
	\includegraphics[width=\linewidth]{q_6}
	\label{fig:q_6}
\end{figure}

\section{SELECT without FROM}

\begin{lstlisting}
SELECT 100 AS Price
,0.07 AS TaxRate
,100 * 0.07 AS TaxAmount /*Eww, magic numbers*/
,100 + (100 * 0.07) AS Total /*Why is SQL gross?*/
\end{lstlisting}\label{sec:q_7}
See figure \ref{fig:q_7}

\begin{figure}[H]\centering
	\caption{Per \ref{sec:q_7}, SELECT custom values into columns w/o a
	FROM clause}
	\includegraphics[width=\linewidth]{q_7}
	\label{fig:q_7}
\end{figure}

\end{document}
