\documentclass{article}

\title{Exercise 1}
\author{Stewart Johnston\\
  {CIS 150 -- Intro to Database Administration}\\
  {NCMC}\\
  {\texttt{johnstons1@student.ncmich.edu}}
}
\date

\begin{document}

\maketitle

\section{Impactful databases}

There are a handful of databases I interact with more or less directly on a regular basis:

\begin{itemize}
	\item User database (Unix)
	\item User database (Windows)
	\item Google's user database
	\item Gmail's database
	\item Steam's game/user database
	\item etc
\end{itemize}

Which is not even to speak of those which I use indirectly, like site databases
which track pages, posts, comments, tags, etc.

Of those listed, impactful is a toss-up between Unix's user system, and gmail's
backend database. I use gmail on a frequent basis; rare is the day that I don't
check it at least twice. I login to my machines, however, more frequently than
that each day.

Once upon a time, Unix-based OSs relied on the \verb|/etc/passwd| file almost
exclusively. The name of the file relates to the utility of the same name. It
has the dubious benefit of being a plaintext file, with no fancy bells or
whistles going into its formatting other than the humble newline and colon.
Each line lists one user's record, and each record's fields are separated by
":" characters. The file strongly resembles a table of rows and columns, and in
this way bears similarity to the relational model so ubiquitous now.

\verb|/etc/passwd| is further similar in the sense that each record starts with
a unique field indicating the username. The list of fields for each user are as
such:

\begin{enumerate}
	\item Username. Must be unique across users listed in the file
	\item Information to validate a user's password. Modern implementations
		almost always store the actual password information in some
		other, secure spot. The encrypted shadow file is one such
		example. This field is usually used to indicate to the OS where
		to find that information.
	\item UID; the user identifier. This does not need to be unique. (One
		multiple-OS workaround is to retool each system's UID for that
		user to match so that files are shareable on one disk.)
	\item GID; the (primary) group identifier. Depending on the umask,
		files created by this user may be initially available to this
		group. (Both this and UID are the primary means by which
		permissions are controlled on files, and are read by the kernel
		for this purpose. ACLs are a more modern implementation which
		allow for more complex permissions than simple UID and GID can
		allow.
	\item The GECOS field. A normally comma-separated list describing, for example, the user's real name, phone number, and so on.
	\item Absolute path to the user's home directory.
	\item Path to the program started every time the user logs in.  
\end{enumerate}

I have born personal witness to this file and its partner named shadow on at
least some linux distributions sharing a lineage with ubuntu and debian. This
archaic method is still functional, certainly, and used. As mentioned,
constructs like Access Control Lists now exist to supplement the functionality
detailed here. The shadow file is itself a similar text-based database, but the
primary fields of its records are encrypted. Shadow has similarly unique
primary key functionality, since by definition it is attached to the username.

However, some modern OSs only use \verb|/etc/passwd| in single-user mode. MacOS
includes such a disclaimer at the top of the file, with commentary to check the
documentation on Open Directory (found in man page \verb|opendirectoryd(8)|).
Single user mode precludes the need for checking standard user login
information, instead assuming that the user is an administrator attempting to
repair software. (Security concerned users should store sensitive information
with encryption, or use whole-disk encryption such as Filvault on MacOS.) As
such, the passwd file now serves the role of listing users created for internal
use by the OS and other software. This is where almighty root may be found.

Exploring the inner workings of the Open Directory daemon and its kin is
outside the scope of this report. However, the presence of such daemons as
launchd and opendirectoryd betray MacOS' BSD lineage.

\end{document}
