%&tex
\documentclass{article}

\usepackage{graphicx}[draft]
\usepackage{color}
\graphicspath{{./img/}}
\usepackage{geometry}
\usepackage{hyperref}
\usepackage{subcaption}
\usepackage{float}
\usepackage{listings}

\title{Exercise 10 -- Database Security and Permissions}
\author{Stewart Johnston\\
  {CIS 150 -- Intro to Database Administration}\\
  {NCMC}\\
  {\texttt{johnstons1@student.ncmich.edu}}
}
\date{\today}

\begin{document}
\lstset{language=SQL,
	basicstyle=\ttfamily,
	keywordstyle=\color{black}\bfseries\underbar,
	floatplacement=tbph}

\maketitle

For your convenience, you will find that most numbers are in fact
hyperlinked references across the document. Likewise, any other URLs
found, such as links to stackoverflow or docs.microsoft, will be
hyperlinks. Unfortunately, I've come to find that the syntax
highlighting that the listings package provides is limited to a smaller
dielect of SQL than is being used here. T-SQL provides many extensions
to the language which are not standard, and it is possible that the
standard which the syntax highlighting is based on an even older dialect
than the most recent standard. Not every keyword is underlined, which is
unfortunate, but it still does 99\% of what I want it to do, so I am not
eager to change anything.

\tableofcontents
\listoffigures

\section{Create a role and assign permissions}
See listing \ref{lst:q_1} and accompanying figure \ref{fig:q_1}
\label{sec:q_1}
\begin{lstinputlisting}[float, label={lst:q_1}, caption={Per
	\ref{sec:q_1}}]{src/q_1.sql}
\end{lstinputlisting}
\begin{figure}[H]\centering
	\caption{Per \ref{lst:q_1}}
	\includegraphics[width=\linewidth]{q_1}
	\label{fig:q_1}
\end{figure}

\section{Create RobertHalliday login and user}
See listing \ref{lst:q_2} and accompanying figure \ref{fig:q_2}
\label{sec:q_2}
\begin{lstinputlisting}[float, label={lst:q_2}, caption={Per
	\ref{sec:q_2}}]{src/q_2.sql}
\end{lstinputlisting}
\begin{figure}[H]\centering
	\caption{Per \ref{lst:q_2}}
	\includegraphics[width=\linewidth]{q_2}
	\label{fig:q_2}
\end{figure}

\section{Dynamically make logins and users}
The task was to programmatically, rather than statically, create logins
and users for every person found in the Administrators table, and then
associate them with the OrderEntry role created in \ref{sec:q_1}.
See listing \ref{lst:q_3} and accompanying figures \ref{fig:q_3}, \ref{fig:q_3-1}, \ref{fig:q_3-2}
\label{sec:q_3}
\begin{lstinputlisting}[label={lst:q_3}, caption={Per
	\ref{sec:q_3}}]{src/q_3.sql}
\end{lstinputlisting}
\begin{figure}[H]\centering
	\caption{Per \ref{lst:q_3}}
	\includegraphics[width=\linewidth]{q_3}
	\label{fig:q_3}
\end{figure}

\begin{figure}[H]\centering
	\caption{Per \ref{lst:q_3}}
	\includegraphics[width=\linewidth]{q_3-1}
	\label{fig:q_3-1}
\end{figure}

\begin{figure}[H]\centering
	\caption{Per \ref{lst:q_3}}
	\includegraphics[width=\linewidth]{q_3-2}
	\label{fig:q_3-2}
\end{figure}

\section{Make a user by hand with the GUI}
\label{sec:q_4}
\begin{figure}[H]\centering
	\caption{Per \ref{sec:q_4}}
	\includegraphics[width=\linewidth]{q_4}
	\label{fig:q_4}
\end{figure}
\begin{figure}[H]\centering
	\caption{Per \ref{sec:q_4}}
	\includegraphics[width=\linewidth]{q_4-1}
	\label{fig:q_4-1}
\end{figure}
\begin{figure}[H]\centering
	\caption{Per \ref{sec:q_4}}
	\includegraphics[width=\linewidth]{q_4-2} \label{fig:q_4-2}
\end{figure}

\section{Remove OrderEntry role}
Users need to be dropped from the role first
See listing \ref{lst:q_5} and accompanying figure \ref{fig:q_5}
\label{sec:q_5}
\begin{lstinputlisting}[label={lst:q_5}, caption={Per
	\ref{sec:q_5}}]{src/q_5.sql}
\end{lstinputlisting}
\begin{figure}[H]\centering
	\caption{Per \ref{lst:q_5}}
	\includegraphics[width=\linewidth]{q_5}
	\label{fig:q_5}
\end{figure}

\section{Make Admin schema}
Also, give Admin schema a table, associate RobertHalliday with it, and
give Robert basic operator permissions for the schema.
See listing \ref{lst:q_6} and accompanying figure \ref{fig:q_6}
\label{sec:q_6}
\begin{lstinputlisting}[float, label={lst:q_6}, caption={Per
	\ref{sec:q_6}}]{src/q_6.sql}
\end{lstinputlisting}
\begin{figure}[H]\centering
	\caption{Per \ref{lst:q_6}}
	\includegraphics[width=\linewidth]{q_6}
	\label{fig:q_6}
\end{figure}

\end{document}
