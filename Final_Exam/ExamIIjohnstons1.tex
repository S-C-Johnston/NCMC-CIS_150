\hypertarget{short-answer}{%
\section{Short Answer}\label{short-answer}}

\begin{enumerate}
\def\labelenumi{\arabic{enumi}.}
\item
  Database schemas are spoken of in two senses. In an abstract formal
  sense, a schema is the definition of the relationships and data
  structures in a database. In the practical sense, a schema is often
  treated by DBMSs and DBAs as a namespace or container for database
  objects, such as tables, indices, etc. Users can be assigned
  permissions to a schema, and by default unless otherwise specified at
  user creation, they use the ``dbo'' schema.
\item
  Views are effectively virtual tables built from stored SELECT
  statements. They are useful for security primarily because permissions
  can be assigned for them such that users can't access the underlying
  data, and because they can represent arbitrary combinations of fields
  to fit the DBA's -- or organization's -- needs, omitting any
  information which is deemed sensitive or irrelevant to the view.
\item
  Database recovery methods include:

  \begin{itemize}
  \item
    Database dumping, using tools or commands built into the DBMS to
    create a SQL script. Running this script will recreate both the
    schema definitions and the data in the DBMS.
  \item
    Filesystem level backups, which require bringing the database
    offline while archives can be made of the files the DBMS uses to
    store the database. These can then be copied and brought back
    online.
  \item
    Transaction or log-based backups. These use a combination of a file
    which can be copied, and the logs of actions performed on the
    database. The file can be brought back online, and an arbitrary
    selection of logs can be replayed on top of that file in
    chronological order. This is useful if there was some destructive
    action performed, and the database needed to be rewound to just
    before this happened.
  \end{itemize}
\item
  If and only if fault tolerance is not a requirement, RAID 0 -- which
  entails striping data across disks -- makes for very fast parallel
  reading and writing of data. It requires only two disks to implement,
  the minimum size for any RAID set, but it is a fragile structure.
  Failure in either disk would make all data unrecoverable.
\end{enumerate}

\hypertarget{longform-questions}{%
\section{Longform questions}\label{longform-questions}}

\hypertarget{security-weaknesses}{%
\subsection{Security Weaknesses}\label{security-weaknesses}}

SQL Server has many moving parts, several of which are not secure. These
include:

\begin{itemize}
\item
  System Administrator and Service users in the database are by default
  granted to/based on accounts with similar roles in the OS itself. If,
  under any circumstance, the box on which the server lives is
  compromised the default behavior also leaves the server open to
  compromise. It is best practice to decouple as much of the server as
  possible from builtin OS accounts. The builtin admin OS account should
  ideally be disabled, if possible, and the users granted SysAdmin
  privileges shouldn't also have admin privileges with the same account
  for the RDBMS. In the same breath, the default access controls for new
  users and logins is quite permissive, and those defaults should be
  changed as soon as possible.
\item
  Many components which serve a real use for SQL Server are also tainted
  with a history of being vulnerable. Services which live on the IIS
  webserver platform are chief among these. The webapps which use the
  database's information, reporting services, and other internet facing
  components should all be installed and run from separate machines,
  virtual or otherwise.
\item
  SQL injection is usually an issue handled outside of the server's
  components, in the back end which receives user input. If all goes
  well, then all input should be sanitized and validated. Programmers
  being people, they are forgetful and make mistakes. This the case, all
  webapp facing procedures should use SQL mechanisms to quote input to
  remove its danger. QUOTENAME() is one such function, although its
  return type is limited.
\end{itemize}

\hypertarget{backup-and-compress-adventureworks}{%
\subsection{Backup and Compress
AdventureWorks}\label{backup-and-compress-adventureworks}}

See figure \ref{fig:q_1}. \label{sec:q_1}

\begin{lstinputlisting}[float, label={lst:q_1}, caption={per \ref{sec:q_1}}]{src/q_1.sql}
\end{lstinputlisting}
\begin{figure}[H]\centering
	\caption{Per \ref{lst:q_1}}
	\includegraphics[width=\linewidth]{q_1}
	\label{fig:q_1}
\end{figure}


\hypertarget{backup-of-six-high-load-dbs}{%
\subsection{Backup of Six High-Load
DBs}\label{backup-of-six-high-load-dbs}}

Given the availability requirements and the low number of modifications
each day, instead of creating full backups each night, I would instead:

\begin{enumerate}
\def\labelenumi{\arabic{enumi}.}
\item
  Create a full backup on the weekend, likely Sunday. SQL Server
  provides this functionality through a command.
\item
  Each day of the week until the next Sunday, after the high load slows
  down, create a differential backup. This functionality is also
  provided through SQL Server commands
\item
  Repeat this process weekly. This can be automated with the use of the
  SQL Server Agent component, or using some OS feature to automate this,
  such as cron in Linux.
\end{enumerate}

\hypertarget{backup-adventureworks-external-drive-for-maintenance}{%
\subsection{Backup AdventureWorks external drive for
maintenance}\label{backup-adventureworks-external-drive-for-maintenance}}

\begin{figure}[H]\centering
	\caption{Per \ref{backup-adventureworks-external-drive-for-maintenance}}
	\includegraphics[width=\linewidth]{q_2-1}
	\label{fig:q_2-1}
\end{figure}

\begin{figure}[H]\centering
	\caption{Per \ref{backup-adventureworks-external-drive-for-maintenance}}
	\includegraphics[width=\linewidth]{q_2-2}
	\label{fig:q_2-2}
\end{figure}

\begin{figure}[H]\centering
	\caption{Per \ref{backup-adventureworks-external-drive-for-maintenance}}
	\includegraphics[width=\linewidth]{q_2-3}
	\label{fig:q_2-3}
\end{figure}

\begin{figure}[H]\centering
	\caption{Per \ref{backup-adventureworks-external-drive-for-maintenance}}
	\includegraphics[width=\linewidth]{q_2-4}
	\label{fig:q_2-4}
\end{figure}

\begin{figure}[H]\centering
	\caption{Per \ref{backup-adventureworks-external-drive-for-maintenance}}
	\includegraphics[width=\linewidth]{q_2-5}
	\label{fig:q_2-5}
\end{figure}

\begin{figure}[H]\centering
	\caption{Per \ref{backup-adventureworks-external-drive-for-maintenance}}
	\includegraphics[width=\linewidth]{q_2-6}
	\label{fig:q_2-6}
\end{figure}

\begin{figure}[H]\centering
	\caption{Per \ref{backup-adventureworks-external-drive-for-maintenance}}
	\includegraphics[width=\linewidth]{q_2-7}
	\label{fig:q_2-7}
\end{figure}


\hypertarget{recover-adventureworks-from-backup}{%
\subsection{Recover AdventureWorks from
backup}\label{recover-adventureworks-from-backup}}

\begin{figure}[H]\centering
	\caption{Per \ref{recover-adventureworks-from-backup}}
	\includegraphics[width=\linewidth]{q_3-1}
	\label{fig:q_3-1}
\end{figure}

\begin{figure}[H]\centering
	\caption{Per \ref{recover-adventureworks-from-backup}}
	\includegraphics[width=\linewidth]{q_3-2}
	\label{fig:q_3-2}
\end{figure}

\begin{figure}[H]\centering
	\caption{Per \ref{recover-adventureworks-from-backup}}
	\includegraphics[width=\linewidth]{q_3-3}
	\label{fig:q_3-3}
\end{figure}

\begin{figure}[H]\centering
	\caption{Per \ref{recover-adventureworks-from-backup}}
	\includegraphics[width=\linewidth]{q_3-4}
	\label{fig:q_3-4}
\end{figure}

\begin{figure}[H]\centering
	\caption{Per \ref{recover-adventureworks-from-backup}}
	\includegraphics[width=\linewidth]{q_3-5}
	\label{fig:q_3-5}
\end{figure}


\hypertarget{disaster-recovery-site-in-the-cloud}{%
\subsection{Disaster Recovery site in the
cloud}\label{disaster-recovery-site-in-the-cloud}}

Best practices for backups of any kind are to have multiple physical
locations where data is stored, both an on-site backup and an off-site
backup. Contracting another nearby company to handle physically
redundant backups would be a lengthy and costly process, not to mention
the installation of equipment to make a space a competent warm or hot
site. Then of course there are physical security needs to be handled.
Using one of a number of secure cloud hosts, such as through Amazon's
Web Services or Microsoft's cloud platform, we can ensure the
confidentiality, integrity, and availability of our data and operations
at relatively low cost.

Cloud service providers generally have very scalable pricing models,
such that unused space and time is not charged for. There is of course
still a charge for idle use and disk storage, but this would be much
lower than the idle cost of a physically redundant site. Configuration
is fast, inexpensive, and easy for most common technology stacks. The
only real bottleneck by comparison is the time to send data over the
wire to the cloud service provider's servers we rent.

Hosting a DR site in the cloud several other benefits not limited to:

\begin{itemize}
\item
  Far-flung remote sites are unlikely to be affected by power outages or
  dangerous weather we may experience locally.
\item
  Duplicating our DR site across availability zones is trivial.
\item
  The service provider already has security procedures in place, both
  physical and digital.
\end{itemize}
