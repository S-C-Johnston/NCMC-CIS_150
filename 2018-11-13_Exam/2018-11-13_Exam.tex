%&tex
\documentclass{article}

\usepackage{graphicx}[draft]
\usepackage{color}
\graphicspath{{./img/}}
\usepackage{geometry}
\usepackage{hyperref}
\usepackage{subcaption}
\usepackage{float}
\usepackage{listings}

\title{Exam on 2018-11-13}
\author{Stewart Johnston\\
  {CIS 150 -- Intro to Database Administration}\\
  {NCMC}\\
  {\texttt{johnstons1@student.ncmich.edu}}
}
\date{\today}

\setcounter{secnumdepth}{3}

\begin{document}
\lstset{language=SQL,
	basicstyle=\ttfamily,
	keywordstyle=\color{black}\bfseries\underbar,
	floatplacement=tbph}

\maketitle

\tableofcontents
\listoffigures

\section{Short answers}

	\subsection{Data independence} comes in two flavors: Physical data
	independence and Logical data independence. The principle function of
	data independence is to allow for changes without having to rewrite
	applications which use the data.  Physical modifications would normally
	be made for performance reasons, but we don't want these changes to
	break queries. Logical changes would happen primarily when the
	conceptual scheme is altered. These might change some constraints, but
	shouldn't interfere with existing operations.

	\subsection{Primary keys} should be both stable and unique. It is
	important that any given row is always able to be referenced
	unambiguously, and that the method of reference never changes. To that
	end, primary keys should be either be made or chosen from immutable
	data. They also cannot be null.


	\subsection{DDL and DML} stand for Data Definition Language and Data
	Manipulation Language, respectively. They are two of the subsets of
	language defined for SQL. Data Definition Language has much to do with
	the creation and alteration of tables, especially of the physical
	data-types used for the information. Data Manipulation Langauge,
	meanwhile, is related to retrieving and editing tuples and records. If
	the initial database design is solid and meets the needs of the users,
	DDL may go untouched for years. DML, on the other hand, will certainly
	see frequent use in any living database.
	
		\begin{lstlisting}[float]
		/* Data Definiton Language example */
		CREATE TABLE Parts(
		id	INTEGER     PRIMARY KEY,
		name	VARCHAR(50) NOT NULL,
		length	FLOAT       NOT NULL,
		width	FLOAT       NOT NULL,
		);
		\end{lstlisting}

		\begin{lstlisting}[float]
		/* Data Manipulation Language example */
		SELECT name, length, width
		FROM Parts
		WHERE length > 12;
		\end{lstlisting}

	\subsection{Foreign Keys} are the primary vehicle for describing and
	constraining relationships in our data. They allow tuples to reference
	other tuples, either in the same or in a different table. They provide
	database administrators a means to keep data consistent, both in the
	form of normalization, and in the form of referential integrity.
	Referential integrity is a constraint for the RDBMS which tells it to
	forbid the deletion of records which are being referenced by other
	records, meaning that to delete a record which is referenced, one must
	prune or re-assign all records from all tables which reference it in
	one shot.

	\subsection{``Three levels of schema'' architecture} involves an
	external schema, conceptual schema, and internal schema. Several
	external schema can exist for any given database, which describe how
	users see the data. Conceptual schema describes the relationships
	between entities in a database. The internal or physical schema describes the
	datatypes and other physical storage requirements for the data.

\section{Hands on with the server}

\subsection{Attaching using SSMS}
\subsubsection{} Right-click on Databases, select Attach... \label{sec:q_1-1-1}
See figure \ref{fig:q_1-1-1}
\begin{figure}[H]\centering
	\caption{Per \ref{sec:q_1-1-1}}
	\includegraphics[width=\linewidth]{q_1-1-1}
	\label{fig:q_1-1-1}
\end{figure}
\subsubsection{} In Databases to Attach, choose Add \label{sec:q_1-1-2}
See figure \ref{fig:q_1-1-2}
\begin{figure}[H]\centering
	\caption{Per \ref{sec:q_1-1-2}}
	\includegraphics[width=\linewidth]{q_1-1-2}
	\label{fig:q_1-1-2}
\end{figure}
\subsubsection{} This opens a file explorer window for browsing, find the database file.
\label{sec:q_1-1-3}
See figure \ref{fig:q_1-1-3}
\begin{figure}[H]\centering
	\caption{Per \ref{sec:q_1-1-3}}
	\includegraphics[width=\linewidth]{q_1-1-3}
	\label{fig:q_1-1-3}
\end{figure}
\subsubsection{} If there is a log file with the database, it will grab that as well. Click
OK.\label{sec:q_1-1-4}
See figure \ref{fig:q_1-1-4}
\begin{figure}[H]\centering
	\caption{Per \ref{sec:q_1-1-4}}
	\includegraphics[width=\linewidth]{q_1-1-4}
	\label{fig:q_1-1-4}
\end{figure}
\subsubsection{} Thusly a database is attached under the master. \label{sec:q_1-1-5}
See figure \ref{fig:q_1-1-5}
\begin{figure}[H]\centering
	\caption{Per \ref{sec:q_1-1-5}}
	\includegraphics[width=\linewidth]{q_1-1-5}
	\label{fig:q_1-1-5}
\end{figure}

\subsection{Attaching a database using DDL}
See figure \ref{fig:q_1-2}
\label{sec:q_1-2}
\begin{lstinputlisting}[float]{src/q_1-2.sql}
\end{lstinputlisting}
\begin{figure}[H]\centering
	\caption{Per \ref{sec:q_1-2}}
	\includegraphics[width=\linewidth]{q_1-2}
	\label{fig:q_1-2}
\end{figure}

\subsection{Entity relationship diagram} showing Primary Keys as `PK\_', foreign
keys with cardinality as crow's foot, and other attributes as part of the
tables. See figure \ref{fig:salesDB_erd}
\label{sec:salesDB_erd}
\begin{figure}[H]\centering
	\caption{Per \ref{sec:salesDB_erd}}
	\includegraphics[]{salesDB_erd}
	\label{fig:salesDB_erd}
\end{figure}

\subsection{Count of customers named ``Smith''}
See figure \ref{fig:q_2}
\label{sec:q_2}
\begin{lstinputlisting}[float]{src/q_2.sql}
\end{lstinputlisting}
\begin{figure}[H]\centering
	\caption{Per \ref{sec:q_2}}
	\includegraphics[width=\linewidth]{q_2}
	\label{fig:q_2}
\end{figure}

\section{Ashley Smith's CustomerID}
See figure \ref{fig:q_3}
\label{sec:q_3}
\begin{lstinputlisting}[float]{src/q_3.sql}
\end{lstinputlisting}
\begin{figure}[H]\centering
	\caption{Per \ref{sec:q_3}}
	\includegraphics[width=\linewidth]{q_3}
	\label{fig:q_3}
\end{figure}

\section{Names and prices of products more than $1000}
See figure \ref{fig:q_4}
\label{sec:q_4}
\begin{lstinputlisting}[float]{src/q_4.sql}
\end{lstinputlisting}
\begin{figure}[H]\centering
	\caption{Per \ref{sec:q_4}}
	\includegraphics[width=\linewidth]{q_4}
	\label{fig:q_4}
\end{figure}

\end{document}
