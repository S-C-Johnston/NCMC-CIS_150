\documentclass{article}

\usepackage{geometry}

\title{Exercise 2}
\author{Stewart Johnston\\
  {CIS 150 -- Intro to Database Administration}\\
  {NCMC}\\
  {\texttt{johnstons1@student.ncmich.edu}}
}
\date

\begin{document}

\maketitle

\begin{enumerate}

	\item Definitions:

\begin{enumerate}
	
	\item Data: Bulk sum of all details. E.g., the customer table in totality is data.
	\item Information: Data which has been filtered or queried in a way to
		answer a particular question. E.g., the sum of new customers
		since the 2017 marketing campaign.
	\item Information Management: The process of collecting information,
		editing records, and retrieving actionable information from
		its means of storage. 
	\item Database: The total collection of all organized tables.
	\item Database System: Server software which makes accessible the
		database files which it stores.
	\item Database Management System: The suite of software built to
		interact with the core system.
	\item Database Administrator: A professional whose job pertains to the
		planning, installation, configuration, design, security, backup
		and data recovery involved in a DBMS. Likely to be more
		involved in the hardware and software aspects of running a DBMS
		than in interacting with the data itself.
	\item Database Analyst: A professional whose job involves mainly the
		retrieval and synthesis of data and information to be useable.
		Likely to be more involved in interacting with the data, its
		needs and design rather than the support of making it
		operational.
	\item Decision Support System: The structured support for the
		management decisions for an organization. Software will take
		the form of a front end to an important array of information
		for that user.
	\item System Effectiveness: A metric of how well the DB model is
		handling the needs of that data and its users. Its reliability
		and applicability. Doing the right thing.
	\item System Efficiency: A metric of how much waste of time and effort
		is sunk into retrieving and using data. Or rather, an inverse
		proportion: something is efficient if it has low waste. Doing
		something the right way, and quickly.
	\item User View: The virtual representation of the data being queried.
		Likely to be a client-side, temporary and volatile model. It is
		decoupled from the normalized, structured handling of
		information in the database itself.
	\item Logical Schema: The properly normalized design of the data
		representation. Defines the categories and relationships in
		play at the center of the system.
	\item Physical Schema: The literal physical implementation of the
		Logical Schema. The exact datatypes which are used by the
		database system, as well as its files, logs, etc.

\end{enumerate}

\item Data is and databases are an important organizational resource for several
reasons. The ability to quickly and reliably find information relevant to daily
operations increases productivity tremendously, rather than wasting time
searching for information. Keeping data in a consistent,
secure way helps to prevent loss or confusion of service to clients and end
users.

\item The main goal of a database system is to securely dispense data or information sought
by its authorized users. A database system would not be effective if it allowed
unauthorized users to access data they shouldn't, nor would it be effective if it failed to reliably return authorized queries.

\item The three levels of the ANSI/SPARC architecture are the Physical/Internal,
Conceptual, and External schema. The External Schema correspond to end user
views of the data they need. The Conceptual Schema describes the information
needs, and its design and structure. The Physical Schema is the storage of that
information by the computer, which spans from the datatype implementation to
the files and logs it keeps.

\end{enumerate}

\end{document}
