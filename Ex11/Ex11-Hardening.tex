\begin{enumerate}
\def\labelenumi{\arabic{enumi}.}
\item
  Install only the required SQL server components

  \begin{enumerate}
  \def\labelenumii{\arabic{enumii}.}
  \item
    The effect of only installing required components is in part to
    reduce the possible attack surface of the Server. Adding other
    components later is in effect trivial, and offers no real benefit if
    installed early. Although the Server may not load those components
    until they are needed, I can't say this for certain. By waiting to
    install additional modules until they're needed, the runtime
    hardware demands are kept as light as possible.
  \item
    I think this is best practice with any technology, from SQL to
    Operating System. Stripping out unnecessary modules from the Linux
    kernel and compiling it customized for the target hardware is not
    uncommon.
  \end{enumerate}
\item
  Don't install SQL Server Reporting Services (SSRS) on the same server
  as the database engine

  \begin{enumerate}
  \def\labelenumii{\arabic{enumii}.}
  \item
    Server Reporting Services run with internet facing pieces on top of
    Microsoft IIS, which has historically not been a bulwark of
    security. Running it on the same server as the DBMS adds enormous
    surface area potentially vulnerable.
  \item
    For the same reasons as the first section, this seems to be a best
    practice which would be wise to apply in all circumstances. Even if
    the server it runs on is not on different hardware, merely a
    different virtual machine, that insulation of internetworked
    components is wise.
  \end{enumerate}
\item
  Disable services not expecting immediate use

  \begin{enumerate}
  \def\labelenumii{\arabic{enumii}.}
  \item
    This may come with some management overhead, but it is probable that
    much of it could be scripted away. This is another property which
    falls out of the first. Keeping both runtime hardware requirements
    and attack surface slim is not bad as long as it doesn't interfere
    with performance.
  \item
    Again, this seems to me like a best practice with few exceptions.
  \end{enumerate}
\item
  Change the default TCP/IP ports

  \begin{enumerate}
  \def\labelenumii{\arabic{enumii}.}
  \item
    This may be helpful insofar as preventing vulnerability to malware
    which is not particularly smart, likewise malicious people who are
    not particularly smart. That may well be enough to keep threats from
    organization machines from being effective.
  \item
    Of the software means to protect a SQL server, this is likely one of
    the weakest. It isn't necessarily a bad thing to do, but it remains
    effective only if the server isn't vulnerable to port sniffing. For
    a low-risk server, changing the defaults may not be worth the
    hassle. The fewer users there are on a network accessing a server
    from organization machines, the less risk there is of malware
    threatening the server.
  \end{enumerate}
\item
  Disable the network protocols not required for operation

  \begin{enumerate}
  \def\labelenumii{\arabic{enumii}.}
  \item
    Another means of keeping the surface area slim, insulating the SQL
    server by running it without any other networking stack on the same
    host is wise.
  \item
    This depends on the availability and difficulty in achieving this.
    If favoring named pipes over TCP/IP means changing significant
    chunks of application logic to connect properly, then it's certainly
    not worth the effort for such databases as we've been running.
  \end{enumerate}
\item
  Keeping anti-malware and firewall software updated and configured
  correctly

  \begin{enumerate}
  \def\labelenumii{\arabic{enumii}.}
  \item
    On its own, keeping firewall and anti-malware software running
    smoothly and correctly on any system will be tremendously effective
    in stopping a significant chunk of threats in their tracks. Firewall
    software which is too permissive, or blocks the incorrect ports,
    could prevent organization operations. Anti-malware software which
    is not updated can't catch any recently discovered threats.
  \item
    Especially for a small business unlikely to face targeted or
    tailored threats, the benefit gained and the relative ease at which
    it is attained moves this into best practice territory, with very
    few exceptions.
  \end{enumerate}
\item
  Manage the surface area configurations

  \begin{enumerate}
  \def\labelenumii{\arabic{enumii}.}
  \item
    Microsoft in particular often does not have what I perceive as sane
    defaults. Using a tool to manage defaults before a server is even
    deployed, and to wrangle the exact policy and configuration after
    deployment is only smart. Being able to touch these configurations
    and following through on that ability is, to my mind, a solid chunk
    of an administrator's role.
  \item
    Many defaults may not require wrangling, especially for
    smaller-scale organizations, those with low risk, or those without
    sensitive information. Managing the surface area configurations will
    still inevitably come up. Even if only one policy sees change, doing
    so from one unified tool and tracking the changes is too easy an
    option to ignore.
  \end{enumerate}
\item
  Configuring authentication

  \begin{enumerate}
  \def\labelenumii{\arabic{enumii}.}
  \item
    The recommended course of action is to use Windows Authentication
    only, whenever possible. In a homogenous environment, this helps
    reduce moving parts and points of ingress. MS SQL Server makes it
    trivial to introduce new logins based on Windows Authentication,
    which is convenient. If the DBA is not wearing an excessive number
    of hats, such as also being the SysAdmin for the network, this can
    help offload the work of security. This comes with the trade of
    having to trust the SysAdmin to have strong security policy. The
    security is only as strong as the weakest link, after all.
  \item
    For a small business, it is likely that the DBA is also the
    SysAdmin, and moving the security responsibility from the DBMS to
    the OS does not also move it outside of the DBA's influence. There
    isn't a compelling logistical reason not to use Windows
    Authentication for all Windows users on the network. The only qualm
    I have with it is that Windows Authentication is not particularly
    difficult to break, especially with access to the box. Still, it is
    a practical choice.
  \end{enumerate}
\item
  Configure administrative accounts

  \begin{enumerate}
  \def\labelenumii{\arabic{enumii}.}
  \item
    Removing the builtin administrator accounts and heavily restricting
    roles and permissions will go a long way to preventing privilege
    escalation and unauthorized tampering.
  \item
    Again, the security is only as strong as the weakest link. Securing
    the server on which the DBMS runs is as important, if not more, than
    securing the DBMS itself. This should be fairly trivial to put in
    place, and seems to me an obvious best practice.
  \end{enumerate}
\item
  Provision service accounts using non-user accounts which are not
  built-in

  \begin{enumerate}
  \def\labelenumii{\arabic{enumii}.}
  \item
    User accounts have permissions on many machines, which makes them a
    more vulnerable vector for attack, so non-user accounts are best.
    Built-in accounts inherit some permissions from the defaults. Using
    custom non-user accounts is wise in combination with the principle
    of least privilege, and the separation of duties. Splitting up
    administration over multiple components of a hardware network is
    recommended for similar reasons.
  \item
    This is mildly finicky, but for a small business, the small number
    of moving parts makes this kind of hardening practice trivial to
    implement. For increasingly sensitive data, this is increasingly
    necessary. Except for the forethought in configuring it, I can see
    no reason not to implement this.
  \end{enumerate}
\item
  Creating security groups

  \begin{enumerate}
  \def\labelenumii{\arabic{enumii}.}
  \item
    Applying roles and permissions to domain security groups makes
    administration of larger networks significantly easier. Rather than
    handling an increasing number of individual members at the DBMS
    level, handling the abstract security groups makes management
    significantly less granular at the DBMS.
  \item
    For fewer than a dozen users, this is probably excessive extra work
    to implement this. Such small organizations as would be using the
    MyGuitarShop database would require so little overhead to just
    handle individual users, that adding the overhead of this
    abstraction doesn't make sense. Only if a small business were
    expecting significant growth would this abstraction be worth setting
    up at this scale.
  \end{enumerate}
\item
  Schema and object ownership

  \begin{enumerate}
  \def\labelenumii{\arabic{enumii}.}
  \item
    Granting ownership in wanton fashion is unwise. The dbo schema is so
    wide-reaching that no individual user should be mapped to own it,
    and when they do need to own something, allocating schemas is fairly
    trivial. Keeping those schemas on a per-database level, rather than
    allowing them to chain, will help prevent unauthorized (if
    unintentional) access to data.
  \item
    Working with schemas is trivial to implement on the DBMS level, and
    there is no compelling reason not to adhere to this advice. This is
    a best practice with few exceptions, as far as I can tell.
  \end{enumerate}
\end{enumerate}
